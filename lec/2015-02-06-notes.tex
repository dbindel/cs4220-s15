\documentclass[12pt, leqno]{article}
\usepackage{amsfonts}
\usepackage{amsmath}
\usepackage{fancyhdr}
\usepackage{hyperref}
\usepackage{tikz}
\usepackage{pgfplots}
\usepackage{listings}

\newcommand{\bbR}{\mathbb{R}}
\newcommand{\bbC}{\mathbb{C}}
\newcommand{\matlab}{\textsc{Matlab}}

\newcommand{\hdr}[2]{
  \pagestyle{fancy}
  \lhead{Bindel, Spring 2015}
  \rhead{Numerical Analysis (CS 4220)}
  \fancyfoot{}
  \begin{center}
    {\large{\bf Notes for #1}}
  \end{center}
  \lstset{language=matlab,columns=flexible}  
}


\begin{document}
\hdr{2015-02-06}

\section*{Stability of LU and iterative refinement}

Gaussian elimination with partial pivoting is almost always backward
stable in practice, {\bf but} there are examples where it fails to be
backward stable.  Remember that backward stability in this case
means that if $\hat L$ and $\hat U$ are the computed factors, then
\[
  P(A+E)-\hat L \hat U, \quad \|E\| \leq C \macheps \|A\|
\]
where $C$ is some modest value that depends polynomially on the size
of $A$.

It is possible to diagnose failure of backward stability by looking
at the quantities appearing during the LU factorization.  But a more
useful trick is to look at the residual error\footnote{%
  Or should it be $A \hat{x}-b$?  Doesn't really matter much, so
  long as I'm internally consistent.  In any event, we're often
  concerned with the magnitude of the residual, and not the direction.
}
\[
  r = b-A \hat x.
\]
where $\hat x$ is an approximate solution computed from the LU factors
that are actually stored in the machine.  Note that
\[
  r = A(x-\hat{x})
\]
and by combining the inequalities
\begin{align*}
  \|\hat x-x\| &\leq \|A^{-1}\| \|r\| \\
  \|b\| & \geq \|A\| \|x\|
\end{align*}
we have
\[
  \frac{\|hat x-x\|}{\|x\|} \leq
  \kappa(A) \frac{\|r\|}{\|b\|}.
\]

Of course, if we compute the residual in ordinary floating point, we
might be concerned that the computed residual mostly consists of
rounding error.  On the other hand, unlike Gaussian elimination, we
have a backward error analysis for matrix multiplication that does not
suffer from the uncomfortable caveat ``it is backward stable except rarely,
when it it isn't.''  Also, the cost of computing the residual is
$O(n^2)$, unlike the $O(n^3)$ cost to compute an LU factorization;
hence, we might be willing to pay a little to compute the residual
with extra precision.

What do we do if we have a factorization with a not-tiny backward
error?  After checking the residual to see that the error is
unacceptably large, we might want a way of fixing the problem.
One method for doing this is {\em iterative refinement}, which relies
on the idea that a mediocre factorization may still provide a lot of
value.  The key to iterative refinement is the observation that if
$\hat{x}$ is an approximate solution, then
\[
  A (x-\hat{x}) = r,
\]
or $x = \hat{x} + A^{-1} r$.  If we replace $A^{-1}$ an approximation
$\hat{A}^{-1}$ that comes from solving the system with an approximate
factorization, we have the fixed point iteration
\[
  x^{(k+1}) = x^{(k)} + \hat A^{-1} (b-Ax^{(k)}).
\]
We have already looked at the analysis of fixed point iterations
in 1D; here, the analysis is not much different.  Subtract the
fixed point equation
\[
  x = x + \hat{A}^{-1} (b-Ax),
\]
and we find
\[
  e^{(k+1)} = (I-\hat A^{-1} A) e^{(k)}.
\]
Taking norms, we find that the rate of convergence of the iteration
depends on $\|I-\hat A^{-1} A\|$.  In most cases, this is small enough
for an approximate factorization that iterative refinement restores
backward stability of the result in one step.
%
Of course, just because the result is backward stable doesn't mean
that the forward error will be small!  Hence, for ill-conditioned
problems it is sometimes a good idea to use iterative refinement in
which the residuals are computed with extra precision.

\end{document}
