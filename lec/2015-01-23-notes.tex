\documentclass[12pt, leqno]{article}
\usepackage{amsfonts}
\usepackage{amsmath}
\usepackage{fancyhdr}
\usepackage{hyperref}
\usepackage{tikz}
\usepackage{pgfplots}
\usepackage{listings}

\newcommand{\bbR}{\mathbb{R}}
\newcommand{\bbC}{\mathbb{C}}
\newcommand{\matlab}{\textsc{Matlab}}

\newcommand{\hdr}[2]{
  \pagestyle{fancy}
  \lhead{Bindel, Spring 2015}
  \rhead{Numerical Analysis (CS 4220)}
  \fancyfoot{}
  \begin{center}
    {\large{\bf Notes for #1}}
  \end{center}
  \lstset{language=matlab,columns=flexible}  
}


\begin{document}
\hdr{2015-01-23}

\section*{Logistics}

\begin{itemize}
\item Please make sure you can access CMS and Piazza
\item PS1 is posted and is due next Wednesday
\end{itemize}

\section*{I lost on Jeopardy!}

These are questions you should know something about from the reading.

\begin{enumerate}
\item
  Suppose you want to find $f(x_*) = 0$, and you find an approximation
  $\hat{x}$.  What's the relation between:
  \begin{enumerate}
  \item Absolute error ($|\hat{x}-x_*|$)
  \item Relative error ($|\hat{x}-x_*|/|x_*|$)
  \item Residual error ($f(\hat{x})$)
  \end{enumerate}
\item
  Why might you sometimes need to use termination criteria that aren't
  based on error measures?
\item
  Suppose $x_*$ is a fixed point of $g(x)$.  Under what condition does
  fixed point iteration on $g$ converge to $x_*$ starting from nearby
  points?
\item
  \matlab\ has a fairly robust function {\tt fzero} that finds the
  zeros of a nonlinear function of one variable (using a combination
  of bisection and secant iteration).  Why might you want to sometimes
  roll your own?
\end{enumerate}

\section*{Choice of functions and variables}

Root-finding problems are hard or easy depending on how they are
posed.  Often, the initial problem formulation is not the most
convenient.  For example, consider the problem of finding the positive
root of
\[
  f(x) = (x+1)(x-1)^8-10^{-8}.
\]
This function is terrifyingly uninformative for values close to $1$.
Newton's iteration is based on the assumption that a local, linear
approximation provides a good estimate of the behavior of a function.
In this problem, a linear approximation is terrible.  Fortunately, the
function
\[
  g(x) = (x+1)^{1/8} (x-1)-10^{-1}
\]
has the same root, which is very nicely behaved.

There are a few standard tricks to make root-finding problems easier:
\begin{itemize}
\item 
  Scale the function.  If $f(x)$ has a zero at $x_*$, so does $f(x)
  g(x)$; and sometimes we can analytically choose a scaling function
  to make the root finding problem easier.
\item
  Otherwise transform the function.  For example, in computational
  statistics, one frequently would like to maximize a likelihood
  function
  \[
    L(\theta) = \prod_{j=1}^n f(x_j; \theta)
  \]
  where $f(x; \theta)$ is a probability density that depends on some
  parameter $\theta$.  One way to do this would be find zeros of
  $L'(\theta)$, but this often leads to scaling problems (potential
  underflow) and other numerical discomforts.  The standard trick is
  to instead maximize the log-likelihood function
  \[
    \ell(\theta) = \sum_{j=1}^n \log f(x_j; \theta),
  \]
  often using a root finder for $\ell'(\theta)$.  This tends to be a
  much more convenient form, both for analysis and for computation.
\item
  Change variables.  A good rule of thumb is to pick variables that
  are naturally {\em dimensionless}\footnote{%
    Those of you who are interested in applied mathematics more
    generally should look up the Buckingham Pi Theorem --- it's a
    tremendously useful thing to know about.  }
  For difficult problems, these dimensionless variables are often very
  small or very large, and that fact can be used to simplify the
  process of coming up with good initial guesses for Newton iteration.
\end{itemize}

\section*{Monitoring convergence}

\begin{figure}
  \begin{tikzpicture}
    \begin{semilogyaxis}[
        width=0.45\textwidth,
        grid={major}
      ]
      \addplot table[x=k,y=g] {2015-01-23-codes/newtong.dat};
      \legend{$g(x_k)$};
    \end{semilogyaxis}
  \end{tikzpicture}
  \hfill
  \begin{tikzpicture}
    \begin{semilogyaxis}[
        width=0.45\textwidth,
        grid={major}
      ]
      \addplot table[x=k,y=dx] {2015-01-23-codes/newtong.dat};
      \legend{$dx_k$};
    \end{semilogyaxis}
  \end{tikzpicture}
  \caption{Convergence of Newton's iteration for
    $g(x) = (x+1)^{1/8} (x-1) - 0.1$}
  \label{fig:convergence}
\end{figure}

A very natural way to monitor convergence is via a semi-logarithmic
plot.  We plot both the residual error $g(x_k)$ and the change
$x_{k+1}-x_k$.  Linearly convergent iterations such as bisection and
simple fixed-point methods should produce linear convergence plots.
Newton is quadratically convergent, which means that the convergence
plot should look like a downward-facing parabola once convergence
kicks in (Figure~\ref{fig:convergence}).  In practice, convergence is
so fast that there are often only two or three points between when the
iteration begins to converge and when floating point errors begin to
dominate.

If you compute a derivative wrong when you code Newton iteration, the
resulting iteration will often converge -- but linearly.  So looking
at a plot like this is a good way to diagnose coding errors that
affect performance without ultimately affecting the ability to compute
a correct result.

Note that I have used the {\tt pgfplots} package for drawing my plots;
this tends to look nice in \LaTeX, but it's not always the simplest
thing.  You are welcome to mimic this file if you want to use {\tt
  pgfplots} to draw convergence plots for your homework; alternately,
you can use \matlab\ graphics.  Or, if you prefer, you can use
Python and {\tt matplotlib}.  We only really need \matlab\ code for
problems where you will submit the code.

\section*{Starting points}

All root-finding software requires either an initial guess at the
solution or an initial interval that contains the solution.  This
sometimes calls for a little cleverness, but there are a few standard
tricks:
\begin{itemize}
\item
  If you know where the problem comes from, you may be able to get a
  good estimate (or bounds) by ``application reasoning.''  This is
  often the case in physical problems, for example: you can guess the
  order of magnitude of an answer because it corresponds to some
  physical quantity that you know about.
\item
  Crude estimates are often fine for getting upper and lower bounds.
  For example, we know that for all $x > 0$,
  \[
    \log(x) \leq x-1
  \]
  and for all $x \geq 1$, $\log(x) > 0$.  So if I wanted to $x +
  \log(x) = c$ for $c > 1$, I know that $c$ should fall between $x$
  and $2x-1$, and that gives me an initial interval.  Alternatively,
  if I know that $g(z) = 0$ has a solution close to $0$, I might try
  Taylor expanding $g$ about zero -- including higher order terms if
  needed -- in order to get an initial guess for $z$.
\item
  Sometimes, it's easier to find local minima and maxima than to find
  zeros.  Between any pair of local minima and maxima, functions are
  either monotonically increasing or monotonically decreasing, so
  there is either exactly one root in between (in which case there is
  a sign change between the local min and max) or there are zero roots
  between (in which case there is no sign change).  This can be a
  terrific way to start bisection.
\end{itemize}

\section*{Problems to ponder}

\begin{enumerate}
\item
  Analyze the convergence of the fixed point iteration
  \[
    x_{k+1} = c - \log(x_k).
  \]
  What is the equation for the fixed point?  Under what conditions
  will the iteration converge with a good initial guess, and at what
  rate will the convergence occur?
\item
  Repeat the previous exercise for the iteration 
  $x_{k+1} = 10-\exp(x_k)$.

\item
  Analyze the convergence of Newton's iteration on the equation 
  $x^2 = 0$, where $x_0 = 0.1$.  How many iterations will it take to get
  to a number less than $10^{-16}$?

\item
  Analyze the convergence of the fixed point iteration
  $x_{k+1} = x_k-\sin(x_k)$ for $x_k$ near zero.  Starting from 
  $x = 0.1$, how many iterations will it take to get to a number 
  less than $10^{-16}$?

\item
  Consider the cubic equation
  \[
    x^3 - 2 x + c = 0.
  \]
  Describe a general purpose strategy for finding {\em all} the real roots
  of this equation for a given $c$.

\item
  Suppose we have some small number of samples $X_1, \ldots, X_m$
  drawn from a Cauchy distribution with parameter $\theta$ (for which
  the pdf is)
  \[
    f(x,\theta) = \frac{1}{\pi} \frac{1}{1+(x-\theta)^2}.
  \]
  The {\em maximum likelihood estimate} for $\theta$ is the function
  that maximizes
  \[
    L(\theta) = \prod_{j=1}^m f(X_j,\theta).
  \]
  Usually, one instead maximizes $l(\theta) = \log L(\theta)$ --- why would this
  make sense numerically?  Derive a MATLAB function to find the
  maximum likelihood estimate for $\theta$ by finding an appropriate
  solution to the equation $l'(\theta) = 0$.

\item 
  The Darcy friction coefficient $f$ for turbulent flow in a pipe is 
  defined in terms of the Colebrook-White equation for large
  Reynolds number $\mathrm{Re}$ (greater than 4000 or so):
  \[
    \frac{1}{\sqrt{f}} = -2 \log_{10}\left(
      \frac{\epsilon/D_h}{3.7} + \frac{2.51}{\mathrm{Re}{\sqrt{f}}}
      \right)
  \]
  Here $\epsilon$ is the height of the surface roughness and $D_h$ is
  the diameter of the pipe.  For a 10 cm pipe with 0.1 mm surface
  roughness, find $f$ for Reynolds numbers of $10^4$, $10^5$, and
  $10^6$.  Ideally, you should use a Newton iteration with a good initial guess.

\item
  A cable with density of 0.52 lb/ft is suspended between towers of
  equal height that are 500 ft apart.  If the wire sags by 50 ft in
  between, find the maximum tension $T$ in the wire.  The relevant
  equations are
  \begin{align*}
    c + 50 &= c \cosh\left( \frac{500}{2c} \right) \\
    T &= 0.52(c + 50)
  \end{align*}
  Ideally, you should use a Newton iteration with a good initial guess.
\end{enumerate}


\end{document}
