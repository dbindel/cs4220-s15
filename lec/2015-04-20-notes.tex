\documentclass[12pt, leqno]{article}
\usepackage{amsfonts}
\usepackage{amsmath}
\usepackage{fancyhdr}
\usepackage{hyperref}
\usepackage{tikz}
\usepackage{pgfplots}
\usepackage{listings}

\newcommand{\bbR}{\mathbb{R}}
\newcommand{\bbC}{\mathbb{C}}
\newcommand{\matlab}{\textsc{Matlab}}

\newcommand{\hdr}[2]{
  \pagestyle{fancy}
  \lhead{Bindel, Spring 2015}
  \rhead{Numerical Analysis (CS 4220)}
  \fancyfoot{}
  \begin{center}
    {\large{\bf Notes for #1}}
  \end{center}
  \lstset{language=matlab,columns=flexible}  
}


\newcommand{\calK}{\mathcal{K}}
\newcommand{\calP}{\mathcal{P}}
\newcommand{\calR}{\mathcal{R}}

\begin{document}
\hdr{2015-04-20}

Not all of these problems would be suitable as exam questions, but
many of the ideas exercised (e.g.~block Gaussian elimination,
understanding convergence from a semi-logarithmic plot of residuals vs
iterations, and writing an error iteration to analyze local
convergence of a fixed point iteration) are things that I would expect
you to be able to do on an exam.

\paragraph*{1: A different iteration}
In project 3, you are asked to solve a system of equations
for an equilibrium position of a electrostatically actuated
cantilever subject to displacement control:
\[
F(u,V^2) =
\begin{bmatrix}
  Ku-V^2 \hat{f_e}(u) \\
  e_{\mathrm{tip}}^T u - d
\end{bmatrix}
\]
Consider the fixed point iteration
\[
  x^{(k+1)} = x^{(k)} - \hat{J}_k^{-1} F(x^{(k)}), \quad x
  \equiv \begin{bmatrix} u \\ V^2 \end{bmatrix}
\]
where
\[
\hat{J}_k =
\begin{bmatrix}
  K & -\hat{f}_e(u) \\
  e_{\mathrm{tip}}^T & 0
\end{bmatrix}.
\]
\begin{enumerate}
\item Why is this not Newton's method?
\item Describe how to compute each iterate with one solve with a
  pre-factored system (assuming the Cholesky factorization of $K$ is
  computed at the outset)
\item Why is the constraint $e_{\mathrm{tip}} u^{(k)} = d$ always
  satisfied for $k > 0$?
\end{enumerate}

{\bf Answer:}
\begin{enumerate}
\item The Jacobian of $F$ has $K-V^2 \partial \hat{f}_e/\partial u$
  in the $(1,1)$ block.  Since our iteration only involves $K$ in the
  $(1,1)$ block, it is not Newton.
\item We can solve the update equation
  \[
  \begin{bmatrix}
    K & -\hat{f}_e \\
    e_{\mathrm{tip}}^T & 0
  \end{bmatrix}
  \begin{bmatrix} \Delta u \\ \Delta V^2 \end{bmatrix} =
  \begin{bmatrix} F_1 \\ F_2 \end{bmatrix}
  \]
  by block Gaussian elimination.  This yields
  \begin{align*}
    w_{\mathrm{tip}} &= K^{-1} e_{\mathrm{tip}} \\
    \Delta V^2 &=
      \frac{F_2-w_{\mathrm{tip}}^T F_1}
           {w_{\mathrm{tip}}^T \hat{f}_e} \\
    K \, \Delta u &= F_1 - \Delta V^2 \, \hat{f}_e
  \end{align*}
  The solve to compute $w_{\mathrm{tip}}$ can be done once in a
  pre-computation phase, and does not need to be recomputed after.
  The solve for $\Delta u$ is the only solve that needs to be
  computed at each iteration, and it can be done with two
  triangular solves using the Cholesky factorization $K = R^T R$.
\item The constraint $e_{\mathrm{tip}} u^{(k)} = d$ is always
  satisfied for $k > 0$ because of the second update equation:
  \[
    e_{\mathrm{tip}}^T \Delta u^{(k)} =e_{\mathrm{tip}}^T u^{(k)}-d.
  \]
  Rearranging terms yiels
  \[
    e_{\mathrm{tip}}^T u^{(k+1)} =  
    e_{\mathrm{tip}}^T \left( u^{(k)}-\Delta u^{(k)} \right) = d.
  \]
\end{enumerate}


\paragraph*{2: Rates of convergence}
Typically, we plot residuals or errors on a semi-logarithmic plot in
order to visually verify convergence behaviors.  How can we
numerically test that $\|r^{(k+1)}\| \approx C \|r^{(k)}\|^q$ by
looking at successive residuals?  If the residual in Newton's method
decreases quadratically, must the error also decrease quadratically?

{\bf Answer:}
The natural thing to do is to look at logarithms.  Under this model,
\[
  \log \frac{\|r^{(1)}\|}{\|r^{(0)}\|} \approx q
  \log \frac{\|r^{(2)}\|}{\|r^{(1)}\|}.
\]
We can also compute the constant $C$ from three successive residuals.

This may be worth doing when looking at a linearly convergent
iteration; for quadratically convergent iterations, there are
typically only one or two steps between when convergence kicks in
and when the iteration is dominated by roundoff.  In this case,
a picture can be much more illuminating than a number.

If the Jacobian at the solution is nonsingular at the solution
to a system of nonlinear equation, then the error close to
convergence is well approximated by $J_*^{-1}$ times the residual,
and so quadratic convergence of the residual implies quadratic
convergence of the error.

\paragraph*{3: Pseudoinverse sensitivity}
Assuming $A \in \bbR^{m \times n}$ is full column rank,
what is the first-order term $L$ in the Taylor expansion
\[
(A+\delta A)^\dagger = A^\dagger + \delta \left[ A^\dagger \right] +
                      O(\|\delta A\|^2)?
\]

{\bf Answer:}
This problem involves two points: picking a useful representation
for the pseudo-inverse, and understanding how to differentiate
matrix-valued functions.  For the representation, recall that
\[
  (A+\delta A)^\dagger =
  \left[ (A+\delta A)^T(A+\delta A) \right]^{-1} (A+\delta A)^T.
\]
Let's define the Gram matrix $G = A^T A$.  By the product rule
\[
  \delta \left[ A^\dagger \right] =
    \delta \left[ G^{-1} \right] A^T + G^{-1} \, \delta A^T.
\]
The derivative of a matrix inverse is
\[
  \delta [G^{-1}] = -G^{-1} [\delta G] G^{-1},
\]
so, together with another application of the product rule, we have
\[
  \delta \left[ G^{-1} \right] =
    -G^{-1} [(\delta A)^T A + A^T (\delta A)] G^{-1}
\]
Putting everything together, we have
\[
  \delta [A^\dagger] =
    G^{-1} (\delta A)^T -
    G^{-1} \left[ (\delta A)^T A + A^T (\delta A) \right] G^{-1}.
\]
This is a bit of a mess, but it actually is something we can use in
computation.

\paragraph*{4: Gauss-Newton}
Show that Gauss-Newton iteration generally converges quadratically if the
residual at the minimum is zero and the Jacobian at the origin is
nonsingular, but linearly if the minimum residual is nonzero.

{\bf Answer:}
The Gauss-Newton iteration is
\[
  x_{k+1} = x_k - J_k^\dagger F_k
\]
where $J_k = \partial F(x_k)/\partial x$ and $F_k = F(x_k)$.
Let $J_*$ denote the Jacobian at the solution $x_*$, let
$x_k = x_* + e_k$, and note that
\begin{align*}
  F_k &= F_* + J_* e_k + O(\|e_k\|^2) \\
  J_k^\dagger &= J_k + E_k + O(\|e_k\|^2)
\end{align*}
where $E_k$ is a complicated expression (see the previous problem).
For our purposes, we mostly just care that $E_k = O(\|e_k\|)$.
In this case, the error iteration for Gauss-Newton is
\[
  e_{k+1} = e_k - (J_*^\dagger + E_k) (F_* + J_* e_k) + O(\|e_k\|^2)
\]
Note that $J_*^\dagger F_* = 0$, $E_k J_* e_k = O(\|e_k\|^2)$,
and $J_*^\dagger J_* e_k = e_k$, so
\[
  e_{k+1} = -E_k F_* + O(\|e_k\|^2)
\]
If $F_*$ is zero, we therefore have quadratic convergence.
Otherwise, convergence is linear.

One of the things that might cause us some concern is the possibility
that Gauss-Newton might diverge.  Indeed, this is a possibility if one
starts far from the solution, but the Gauss-Newton direction {\em is}
a descent direction.  That is, looking at the Gauss-Newton direction
$p_k = -J_k^\dagger F_k$ together with the gradient of the the
objective function $\nabla \nabla = J_k^T F_k$ ($\phi = \|F\|^2/2$),
we have
\[
  p^T \nabla \phi = -p^T (J_k^T J_k)^{-1} p.
\]
and if the Jacobian is nonsingular at the solution point, then
$J_k^T J_k$ must be positive definite for points sufficiently nearby.

\end{document}
