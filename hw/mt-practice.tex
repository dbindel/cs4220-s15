\documentclass[12pt, leqno]{article}
\usepackage{amsthm}
\usepackage{amsfonts}
\usepackage{amsmath}
\usepackage{fancyhdr}
\usepackage{hyperref}
\usepackage{tikz}
\usepackage{pgfplots}
\usepackage{listings}

\newcommand{\bbR}{\mathbb{R}}
\newcommand{\bbC}{\mathbb{C}}
\newcommand{\matlab}{\textsc{Matlab}}

\newcommand{\hdr}[2]{
  \pagestyle{fancy}
  \lhead{Bindel, Spring 2015}
  \rhead{Numerical Analysis (CS 4220)}
  \fancyfoot{}
  \begin{center}
    {\large{\bf Notes for #1}}
  \end{center}
  \lstset{language=matlab,columns=flexible}  
}


\begin{document} \phdr{Practice Midterm}

This is around the right scale and form for a midterm for 4220, but it may
be somewhat harder or easier than the actual midterm.  The actual exam
will be open book and notes, but limited to 50 minutes.  In evaluating
yourself, you may want to try the exam under those conditions.

\paragraph*{1: True/False}
\begin{enumerate}
\item
  Suppose $Ax = b$ and $(A+E) \hat{x} = b$.
  Then $\|\hat{x}-x\| \leq \kappa(A) \|E\|$.
\item
  If $a$ and $b$ are normalized floating point numbers and $a+b$ is in
  the range of normalized floating point numbers, then
  $\fl(a+b) = (a+b)(1+\delta)$ where $|\delta| \leq \macheps$.
\item
  Newton's iteration is quadratically convergent for $f(x) = x^2 = 0$
  for starting points sufficiently near zero.
\item
  If $A$ is singular, Gaussian elimination cannot compute $PA = LU$.
\item
  In Gaussian elimination with partial pivoting, all elements of $L$
  below the main diagonal have magnitude at most one.
\end{enumerate}

\paragraph*{2: Fixed point fandango}
Consider the iteration
\[
  x_{k+1} = 10-\exp(x_k).
\]
The iteration has a fixed point $x_* \approx 2.0706$.  For $x_0$
close enough to $x_*$, does the iteration converge?  Explain
by writing an error recurrence.

\paragraph*{3: Norm!}
The {\em Frobenius norm} of a matrix $A$ is
\[
  \|A\|_F = \sqrt{ \sum_{i,j} a_{ij}^2 }.
\]
Show
\begin{enumerate}
\item
  The Frobenius norm is not an operator norm (hint: consider
  $\|I\|_F$).
\item
  The Frobenius norm is consistent with the two norm, i.e.
  \[
    \|Av\|_2 \leq \|A\|_F \|v\|_2.
  \]
  {\em Hint:} The Cauchy-Schwarz inequality states $|x \cdot y| \leq
  \|x\|_2 \, \|y\|_2$.
\end{enumerate}

\paragraph*{4: Pseudoinverse}
Suppose $A \in \bbR^{n \times m}$ has full column rank, $n > m$.
\begin{enumerate}
\item
  Write the pseudoinverse in terms of $A$, the economy QR
  factorization of $A$, and the economy SVD of $A$.
\item
  Show that if $n = m$, then the pseudoinverse is the same as the
  inverse.
\item
  Give a {\em brief} geometric characterization of the null space of
  $A^\dagger$.
\end{enumerate}

\paragraph*{5: Elimination and low rank}
Consider the matrix
\[
  A = I + uv^T
\]
where $\|u\|_1 < 1$ and $\|v\|_1 < 1$.
\begin{enumerate}
\item
  $A$ must be diagonally dominant.  Briefly state why.
\item
  Show that after one step of Gaussian elimination,
  the Schur complement has the form
  \[
    S = I + \alpha u_2 v_2^T.
  \]
  Write a simple expression for the coefficient $\alpha$.
\end{enumerate}

\end{document}
