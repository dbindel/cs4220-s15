\documentclass[12pt, leqno]{article}
\usepackage{amsthm}
\usepackage{amsfonts}
\usepackage{amsmath}
\usepackage{fancyhdr}
\usepackage{hyperref}
\usepackage{tikz}
\usepackage{pgfplots}
\usepackage{listings}

\newcommand{\bbR}{\mathbb{R}}
\newcommand{\bbC}{\mathbb{C}}
\newcommand{\matlab}{\textsc{Matlab}}

\newcommand{\hdr}[2]{
  \pagestyle{fancy}
  \lhead{Bindel, Spring 2015}
  \rhead{Numerical Analysis (CS 4220)}
  \fancyfoot{}
  \begin{center}
    {\large{\bf Notes for #1}}
  \end{center}
  \lstset{language=matlab,columns=flexible}  
}


\begin{document}
  \pagestyle{fancy}
  \lhead{NetID:}
  \rhead{Numerical Analysis (CS 4220)}
  \fancyfoot{}
  \begin{center}
    {\large{\bf Practice Final}}
  \end{center}
  \lstset{language=matlab,columns=flexible}

The following problem set has 20 sub-problems.  This is likely too
long for two hours, but see how much you can do -- most of the problems
have simple answers.  In the exam, you may use any written material you want
(textbook, notes, review notes, assignments and solutions, etc),
but not a friend or the internet.

\paragraph{1: About rates}
Consider the iteration
\[
  x_{k+1} = x_k + \cos(x_k).
\]
\begin{enumerate}
\item What are the fixed points?
\item Classify each fixed point as attractive or repulsive.
\item For the attractive fixed points, what is the rate of
  convergence?
\end{enumerate}

\paragraph{2: Rewrite for accuracy}
Consider $f(x) \equiv \sqrt{1+x}-\sqrt{1-x}$ when $x \ll 1$.
Write a routine to compute $f$ without catastrophic cancellation.

\paragraph{3: Q-less QR}
MATLAB's sparse QR solver computes a ``$Q$-less'' QR decomposition,
i.e. a sparse $R$ is computed explicitly but not the dense $Q$ factor.
Given a sparse triangular $R$ and a sparse $A$, describe:
\begin{enumerate}
\item
  How would one efficiently solve a least squares problem involving $A$?
\item
  Solves with Q-less QR iteration are less well-behaved than those
  involving standard QR, and so a typical implementation will do a
  step of iterative refinement.  Write a MATLAB fragment to describe
  iterative refinement in this setting.
\end{enumerate}

\paragraph{4: Funky fill}
Consider an SPD matrix with the nonzero pattern
\[
A =
\begin{bmatrix}
  \times &        &        & \times \\
         & \times & \times & \times \\
         & \times & \times &        \\
  \times & \times &        & \times
\end{bmatrix}.
\]
What is the nonzero structure of the Cholesky factor?

\paragraph{5: Frobenius}
Write a line of MATLAB to minimize $\|XA-B\|_F^2$
where $X \in \bbR^{n \times m}$ is unknown and
$A \in \bbR^{m \times p}$ and $B \in \bbR^{n \times p}$
are given, with $m < p$.

\paragraph{6: Residual}
Suppose $r = Ax-b$ is the residual in a least squares problem.
Given only $b$ and the $Q$ factor in an economy QR decomposition
of $A$, what is $\|r\|$?

\paragraph{7: Constrained LS}
Write a routine to minimize $\|Ax-b\|^2$ subject to
$\sum_{j=1}^n x_j = 1$.

\paragraph{8: Transfer trouble}
Suppose $A = Q H Q^T$ is upper Hessenberg.  Argue that
the Hessenberg form provides a way of computing the transfer
function $h(s) = c^T (A-sI)^{-1} b$ in $O(n^2)$ time for arbitrary
$s$.  Give MATLAB code; you may assume backslash with a Hessenberg
matrix requires $O(n^2)$ time (because it does!) -- explain why.

\paragraph{9: Diagonal decisions}
Consider the block $2 \times 2$ matrix
\[
  M = \begin{bmatrix} A & B \\ B & D \end{bmatrix}
\]
where $A, B, D \in \bbR^{n \times n}$ are orthogonal.
Describe an efficient algorithm for computing
all the eigenvalues of $M$.

\paragraph{10: Jacobi jumble}
What is the rate of convergence of Jacobi on a diagonal matrix?

\paragraph{11: Killer Krylov}
Suppose $\hat{A} = A + uv^T$.  Given $A^{-1}$ as a preconditioner,
how many steps of a Krylov subspace method are required to solve
a system with $\hat{A}$, and why?

\paragraph{12: Line search}
Does the Armijo condition for a line search guarantee that
$\|x_{k+1}-x_*\| < \|x_k-x_*\|$?  Why or why not?

\paragraph*{13: Simple solver}
Suppose $(A + \eta \operatorname{diag}(x)) x = b$.  Under what
conditions does the fixed point iteration
\[
  A x_{k+1} = b - \eta \operatorname{diag}(x_k) x_k
\]
converge?  Give a bound on the rate of convergence.

\paragraph{14: Differential deal}
Suppose $H$ is positive semi-definite, and consider the trust region
step
\[
  (H + \lambda I) p = -\nabla \phi.
\]
show that at $\lambda = 0$, $dp/d\lambda = H^{-2} \nabla \phi$.

\paragraph{15: Modified Gauss-Newton}
Consider the Gauss-Newton-like iteration
\[
  p_k = M^{-1} J(x_k)^T r(x_k)
\]
where $M$ is a fixed positive definite matrix that we hope
approximates the matrix $J^T J$ at the minimizer of $\phi(x) =
\|r(x)\|^2/2$.
\begin{enumerate}
\item Write a short MATLAB code to implement the iteration
  efficiently.  You may take $O(m n^2)$ setup time, but you should
  only require $O(mn)$ time per step.
\item Show that $p_k$ is a descent direction (so the iteration will
  converge with line search).
\item Give conditions on $M$ and $r$ such that the iteration is guaranteed
  to be locally convergent without line search.  You may assume
  the Jacobian satisfies a Lipschitz condition
  $\|J(x)-J(y)\| \leq \gamma \|x-y\|$.
\end{enumerate}

\end{document}
