\documentclass[12pt, leqno]{article}
\usepackage{amsthm}
\usepackage{amsfonts}
\usepackage{amsmath}
\usepackage{fancyhdr}
\usepackage{hyperref}
\usepackage{tikz}
\usepackage{pgfplots}
\usepackage{listings}

\newcommand{\bbR}{\mathbb{R}}
\newcommand{\bbC}{\mathbb{C}}
\newcommand{\matlab}{\textsc{Matlab}}

\newcommand{\hdr}[2]{
  \pagestyle{fancy}
  \lhead{Bindel, Spring 2015}
  \rhead{Numerical Analysis (CS 4220)}
  \fancyfoot{}
  \begin{center}
    {\large{\bf Notes for #1}}
  \end{center}
  \lstset{language=matlab,columns=flexible}  
}


\newtheorem{lemma}{Lemma}

\begin{document} \phdr{Proj 1: Review Directions}

Please read and comment on the two project submissions your group has
been assigned to review.  Your reviews should point out any flaws that
you see, but please remain constructive.  And point out
strengths as well!

While the review process is meant to provide groups with feedback,
it is not meant to be an undue burden on the reviewers.  If you cannot
understand what's going on in a particular argument, feel free to say
so and move on -- just try to distinguish in your review whether the
issue is that you found the solution unclear or that you found an
error in the solution.

In your reviews, please point out any tasks for which you can see that
the submission is incomplete or incorrect.  I provide a reference
(expensive) computation for comparison against the computations
in Tasks 4--6.  In your review, please comment on whether you see any
evidence that the submission includes computational sanity checks.

When submitting your reviews on CMS, please make sure that you
submit them in the correct order (i.e.~my first request matches
your {\tt review1.pdf}).  Otherwise, your review may end up being
sent to the wrong group.

\paragraph*{Task 1}
Is the time consistent with what you saw?  The time may vary depending
on whether $e_1$ is formed as a dense vector or a sparse vector; the
latter will be slower than the former.

\paragraph*{Task 2}
\begin{enumerate}
\item
  Did the call to {\tt lu} have four output arguments?  If not, this is
  doing LU without column pivoting.  Similarly, note that any calls to
  {\tt chol(N)} that you see should have three output arguments, i.e.
  {\tt [R,p,S] = chol(N)}.
\item
  Is there an argument about why $PQ = I$?  Are you convinced?
\end{enumerate}

\paragraph*{Task 3}
There are several ways to argue that $N$ is positive definite.  Is the
argument made in this project convincing and free of algebra errors?

\paragraph*{Task 4}
Is the code correct?  In particular, you should check to see that
\begin{enumerate}
\item
  The solve involves no matrix-matrix products.  If all is done
  correctly, every operation should involve a matrix and a vector.
  Remember order of operations!
\item
  The computation with the factors should be much faster than the
  computation in Task 1 (not including the time to compute the
  factorization).
\item
  The value computed with the given expression matches the value from
  Task 1, up to rounding error.  That is, if {\tt xfac} is the value
  computed with the factorization and {\tt x} is the value from Task
  1, then {\tt norm(x-xfac)/norm(x)} should be on the order of a few
  times $10^{-16}$.
\end{enumerate}

\paragraph*{Task 5}
\begin{enumerate}
\item
  Is the update written in the form $\hat{N} = N + UCV^T$ where
  $U, V \in \bbR^{n \times 2}$ and $C \in \bbR^{2 \times 2}$?
  Alternately, is there an update to $\hat{M}$ of this form?
\item
  Does the update correctly account for all elements that have
  changed?  Please sanity check this against recomputing $\hat{N}$
  or $\hat{M}$ from scratch based on $\hat{A}$ for the edge $(1,2)$,
  which you can compute with the code
  \begin{lstlisting}
    hatA = A;
    hatA(1,2) = 0;
    hatA(2,1) = 0;
  \end{lstlisting}
\end{enumerate}

\paragraph*{Task 6}
\begin{enumerate}
\item
  Does the code compute the solution for the modified problem?  As in
  the sanity check for Task 5, it will be useful to compare against
  a reference computation (e.g.~one where the edge $(1,2)$ is
  removed).
\item
  Does the code run significantly faster than the original
  factorization?
\item
  Is there some indication of how the code can be run in stages
  (i.e. a precomputation independent of $(a,b)$, and a relatively
  cheap computation depending on $(a,b)$)?
\end{enumerate}

\paragraph*{Task 7} (Extra credit)
Can you follow the logic in the bound?

\paragraph*{Task 8} (Extra credit)
Does the computation appear reasonable?  Can you reproduce any claimed
results by running the code?


\end{document}
